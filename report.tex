\documentclass[remplissage]{karlito}
%\documentclass[remplissage, impression]{karlito}

\usepackage{karlito}



\title{Titre du projet}
\author{karlito}



\begin{document}

\maketitle
\sommaire


\chapter{Introduction}





\chapter{conclusion}



\tabledesfigures
\appendix



\end{document}




%%%%%%%%%%%%%%%%%%%%%%%%%%%%%%%%%%%%%%%%%%%%%%%%%%
%												%
%		Aide à l'utilisation de ce style			%
%												%
%%%%%%%%%%%%%%%%%%%%%%%%%%%%%%%%%%%%%%%%%%%%%%%%%%
%
%Voici une liste des commandes disponibles dans ce
%style ainsi qu'une explication sur chacunes
%d'entre elles sur comment les utiliser :
%
%


%\chapitregauche{#1} : commence un nouvea chappitre sur la prochaine page de gauche.
%paramètre #1 : nom du chapitre.



%\lstinputlisting[language=c, linerange=debut-fin]{#1} : insert le morceau de programme C se trouvant entre les balises debut et fin.
%paramètre #1 : chemin vers le fichier de code.


%Exemple pour insérer du code directement dans le document : 
%	\begin{lstlisting}[language=c]
%		Put your code here.
%	\end{lstlisting}





%\image{#1}{#2}{#3} : insert l'image dans le document.
%paramètre #1 : chemin vers le fichier image.
%paramètre #2 : nom de la figure à afficher. (sera aussi le label de l'image)
%paramètre #3 : propriété de la figure telle que : width=5cm, ou height=7cm.



%Exemple d'insertion de trois images cote à cotes avec label et titres : 
%	\begin{figure}[ht!]
%		\centering
%		\subfloat[titre1]{\includegraphics[height=5.5cm]{chemin_image.png}}
%		\hspace{1mm}
%		\subfloat[titre2]{\includegraphics[height=5.5cm]{chemin_image.png}}
%		\hspace{1mm}
%		\subfloat[titre3]{\includegraphics[height=5.5cm]{chemin_image.png}}
%		\hspace{1mm}
%		\caption{Titre global}
%		\label{Label global}
%	\end{figure}



%\refer{#1} : permet de faire référance à une image, exemple : (cf figure 12 page 3).
%paramètre #1 : nom du label de l'image.




%Le module de création de boites :
%	\begin{boite}[options]
%		message à afficher dans la boite.
%	\end{boite}
%
%les options disponibles sont :
%- backcolor			: la couleur du font de la boite.
%- bordercolor		: la couleur des contours de la boite.
%- titlecolor		: la couleur du titre de la boite.
%- backtitlecolor	: la couleur du font du titre de la boite.
%- title				: le titre de la boite.
%- colortext			: la couleur du text dans la boite.



%\commande{#1} : affiche un text dans le style d'un terminal d'ordinateur.
%paramètre #1 : text à afficher.
